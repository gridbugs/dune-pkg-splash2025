\documentclass{article}
\usepackage{authblk} % used to generate the list of authors with afiliations
\usepackage{multicol} % gives control over the number of columns
\usepackage{lipsum} % generate lipsum test while experimenting with formatting
\usepackage[margin=1in]{geometry} % use smaller margins than the default
\usepackage{listings}
\usepackage{url}

\title{\bf OCaml Package Management with (only!) Dune}
\date{2025-06-26}
\author[1]{Stephen Sherratt}
\author[1]{Marek Kubica}
\author[1]{Rudi Grinberg}
\affil[1]{Tarides}

\begin{document}
    \maketitle

    \begin{abstract}
        The OCaml build system Dune keeps track of a project's dependencies on
        external software packages. Historically however, Dune has been unable
        to download or install these packages completely independently, relying
        on additional tools to perform these functions. This complicated the
        development of projects in OCaml as users needed to be fluent in both
        Dune as well as an additional tool (often opam) to manage a project's
        dependencies. Recent work on Dune has added package management
        capabilities directly to the build system, enabling workflows where Dune
        is the only tool necessary to develop software in OCaml.

        This talk will showcase Dune's new package management features by
        developing an OCaml program starting from a bare system with no OCaml
        tooling installed.
    \end{abstract}

    \begin{multicols}{2}

        \section{Dune and Opam}

        Dune\footnote{\url{https://github.com/ocaml/dune}} is a build system for OCaml
        projects. It creates executables and libraries by compiling and linking
        OCaml source code from multiple files, tracking inter-file dependencies
        to allow for incremental recompilation. Dune knows how to locate OCaml
        libraries external to a project by understanding the
        \textit{findlib}\footnote{\url{https://projects.camlcity.org/projects/findlib.html}}
        file formats. Dune allows a project to be organized into one or more
        \textit{packages} using a packaging scheme that approximates that of the
        \textit{opam}\footnote{\url{https://opam.ocaml.org}} package manager. This is
        no coincidence; Dune can generate metadata files for packages in the
        format expected by opam to simplify the process of publishing packages
        from Dune projects on opam's default package
        repository\footnote{\url{https://github.com/ocaml/opam-repository}}. Dune's
        own metadata files enumerate the packages depended upon by the project
        for this reason.

        Opam metadata files generated by Dune have historically served a second
        purpose: to allow developers of a Dune project to use opam to
        install all of the software packages necessary to build their
        project locally. Dune's reliance on opam as its de facto standard
        package management tool has led to confusion among developers, in
        particular newcomers from other ecosystems where a single tool
        manages both source tree builds and dependency installation. Dune
        and opam have different user interfaces, file formats and require
        different mental models so familiarity with one tool does not
        contribute to one's understanding of the other. The functionality
        of both tools overlaps, since by merit of opam being a source-based
        package manager, it must also be able to build OCaml projects,
        exacerbating users' confusion about the relationship between the
        two tools. Thus it is rare to find someone with sufficient fluency
        in both Dune and opam to use the two tools in tandem to smoothly
        setup and maintain a project's development environment, even among
        the developers of these tools.

        \section {Dune Package Management}

        Due to the fact that Dune's project metadata files already enumerate
        the opam packages necessary to build a project (historically for the
        purpose of generating a corresponding opam metadata file) and that Dune
        is already capable of locating external libraries used in a project, an
        opportunity presented itself to retrofit package management into Dune in
        such a way that existing Dune projects can take advantage of without
        modification. This work commenced in 2023 and is now in a usable state,
        where it's possible to install Dune from a \textit{binary
        distribution}\footnote{\url{https://preview.dune.build}} and develop OCaml
        projects with no installation of opam or the OCaml compiler being
        necessary.

        Indeed we envision a development workflow where the \textit{only}
        globally-installed OCaml tool is Dune, installed from its binary
        distribution or by a system package manager. Any additional
        OCaml-related software, be it a library, the OCaml compiler itself or
        developer tooling such a code formatter or LSP server, will be installed
        and \textit{invoked} by Dune. This simplifies the developer experience
        for OCaml by making the \texttt{dune} command a common entry point for all
        tasks.

        The following listing shows a terminal session building a project using
        Dune package management:
        \begin{lstlisting}
$ dune pkg lock
$ dune build
        \end{lstlisting}

        The first command computes the transitive closure of the project's
        dependencies, creating or updating a copy of this information within the
        project. This command only needs to be run after the project's
        dependencies change. The second command builds the project, first
        downloading and building its dependencies as necessary.

        Opam still continues to play a role, albeit a less user-visible one: As
        the opam package repository contains a rich set of OCaml packages it
        still is a good location to submit and load packages from. Thus it
        remains the default repository Dune uses when resolving and installing
        packages.

        \section {Reproducible Builds}

        Packages in the opam repository conventionally do not constrain the
        upper bound of versions of their dependencies. Within the opam
        repository itself, a manual process prevents API changes from breaking
        released packages by \textit{adding} constraints where necessary when
        new packages are released. However these additions are rarely
        backported to the development branches of projects, and projects that
        are not released to the opam repository are not protected from
        breakages due to API changes if they don't adopt a similar policy.

        Dune addresses this problem with the introduction of \textit{lock
        directories}. The command \texttt{dune pkg lock} computes the transitive
        dependency closure of the project, including \textit{concrete} version
        numbers for each dependency, and this information is stored in files in
        a directory (typically \texttt{dune.lock}) which can be checked into
        version control and thus shared between the contributors to a project.
        This makes it very likely that if a project builds on one machine, then
        it will build on another machine, and freezes the versions of all
        dependencies, preventing unexpected API changes from breaking the
        project as new package releases are published.

        \subsection {Portable Lock Directories}

        Opam packages may have different dependencies depending on properties of
        the machine where they will be installed, such as its CPU architecture
        or operating system. In order for it to be safe to check lock
        directories into version control, it must be the case that generation of
        the lock directory is agnostic to the machine where the lock directory
        is being generated. Otherwise collaborators on a project with different
        types of system will not be able to share a single checked-in lock
        directory effectively.

        Dune's dependency solver is currently unsophisticated with regard to
        generating portable lock directories, opting to solve the entire
        dependency problem once for each of a prescribed list of platforms, and
        then merging the results. The list of platforms can be controlled by an
        entry in a config file. We find that in practice the performance cost of
        this repeated work is acceptable. We may investigate alternative
        approaches to portable lock directories in the future to
        generate completely generic solutions, such as the solver algorithm
        employed in Python's \textit{uv}
        \footnote{\url{https://github.com/astral-sh/uv/blob/main/docs/reference/resolver-internals.md}}
        package manager.

        \section {Installing the Compiler}

        In the opam ecosystem the OCaml compiler is a mostly-regular package,
        as the system is designed in a way where other languages could be
        supported as well. However, if the OCaml compiler is necessary when
        building an opam package (as is most often the case), the compiler should be
        among that package's dependencies.

        The OCaml compiler is implemented in such a way that the absolute path
        to its install location is included in its executable. This means that
        once installed, the compiler's executable and related files cannot be
        moved to another location. This poses a particular problem for Dune.

        Dune maintains a shared cache of built artifacts that can be reused
        across projects to speed up builds. For artifacts to be safely cached,
        it's necessary that the only files consumed when generating them are
        their stated build inputs, and not other files from the project or wider
        file system. To this end Dune builds each artifact (including packages)
        in a transient sandbox environment where only the artifact's
        dependencies are present, before moving the artifacts into the final
        directory for built artifacts (and also copying them into the shared
        cache). The compiler cannot be built in this manner because the paths
        included in its executable would refer to the transient sandbox where
        the compiler was built.

        As a workaround, Dune treats the compiler package specially, patching
        its package metadata at build-time so that it is installed to a location
        within the user's home directory (typically
        \texttt{\$HOME/.cache/dune/toolchains}). Multiple versions of the
        compiler can be co-installed to this directory, and Dune reuses
        compilers when possible rather than building them anew.

        It's ongoing work to make it possible to relocate the compiler after
        installation, after which point this workaround can be removed from
        Dune.

        \section {Developer Tools}

        Developers typically require tools besides the build system and compiler
        to be productive. Dune contains a mechanism for installing and running
        developer tools. Each developer tool corresponds to an opam package, and
        Dune installs tools using the same package management mechanism as for
        regular project dependencies.

        However, unlike opam the build tools are not installed in the same
        environment, thus the development tool dependencies do not have to be
        compatible with the dependency cone of the project itself. This is a
        major improvement over the previous opam workflow where the
        dependencies of the development tools sometimes created issues for
        projects as both would depend on different versions of the same
        package.

        \section {System Packages}

        Opam contains a mechanism for installing non-opam packages using the
        system's package manager (\texttt{apt}, \texttt{brew}, \texttt{cygwin},
        etc). An opam package may declare an external dependency
        (\textit{depext}) by specifying the name of the external package in as
        many package ecosystems as possible, as packages may be referred to by
        different names by different package managers. Opam knows how to
        interface with many package managers to install depexts automatically.

        Dune is not capable of installing depexts directly, however a project
        can be queried for a list of all depexts among its transitive dependency
        closure. Dune detects which system it is running on when computing
        this list so that depext names can be tailored to be compatible with
        the current machine.

        \section {Future Work}

        There are several features currently missing from Dune's implementation
        of package management that either prevent its use for some projects, or
        require an installation of opam:

        \begin{itemize}
            \item \textbf{Dune cannot build projects with circular dependencies
                via test-only dependencies.} This affects a small number of
                widely used low-level packages including Dune itself. Dune uses
                several external packages for its tests (such as
                \textit{ppx\_expect}\footnote{\url{https://github.com/janestreet/ppx_expect}}),
                and some of these packages depend on
                Dune. This should be a benign circular dependency as Dune does
                not require these packages at build-time, however due to
                implementation details Dune cannot currently handle such a case.
            \item \textbf{Portable lock directories are disabled by default.}
                Dune's default behaviour is currently to generate lock
                directories that are specialized to the platform where they were
                generated, making them unsafe to check into version control.
                Work is ongoing to stabilize this feature so it can be enabled
                by default.
            \item \textbf{Dune lacks commands for querying package
                repositories}. The only way to search for a package or print out
                a package's metadata from the command-line is to use \texttt{opam}.
        \end{itemize}

    \end{multicols}
\end{document}
